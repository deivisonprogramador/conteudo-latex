\documentclass[a4paper,12pt]{article}
\usepackage[utf8]{inputenc}
\usepackage[lmargin=3cm,tmargin=3cm,rmargin=2cm,bmargin=2cm]{geometry}
\usepackage[onehalfspacing]{setspace}
\usepackage[brasil]{babel}
\usepackage{enumerate}
\newcounter{questao}
\setcounter{questao}{1}

\newcommand{\quest}[1]{
\vspace{.5cm}
\noindent \textbf{Questão \thequestao. [{\small #1 pnt}] \stepcounter{questao}}
}
\everymath{\displaystle}

\newcommand{\opt}[5]{
\begin{enumerate}[(a)]
\item #1
\item #2
\item #3
\item #4
\item #5
\end{enumerate}
}

\begin{document}

\quest{1,5} Seja uma função afim $f(x)$, cuja a forma é $f(x)=ax+b$, com $a\in \mathbf{R}, b\in \mathbf{R}$ e $a \neq 0.$ Se $f(-3) = 3$ e $f(3)= -1$, os valores de $\mathbf{a}$ e $\mathbf{b}$, são respectivamente: \bigskip

\noindent $(\ \ ) \ 2 \textit{ e } 9$ 
\hfill$(\ \ ) \ 1 \textit{ e } -4$
\hfill$(\ \ ) \ \frac{1}{3} \textit{ e } \frac{3}{5}$
\hfill$(\ \ ) \ 2 \textit{ e } -7$
\hfill$(\ \ ) \ \frac{-2}{3} \textit{ e } 1$


\quest{1,0} Um economista observa os lucros das
empresas $A$ e $B$ do primeiro ao quarto mês de atividades e chega à conclusão que, para este período, as equações que relacionam o lucro, em reais, e o tempo, em meses, são $L_{A}(t) = 3t-1$ e $L_{B}(t) = 2t+9$. Considerando-se que essas equações também são válidas para o período do quinto ao vigésimo quarto mês de atividades, o mês em que as empresas terão o mesmo lucro será o 

\opt{vigésimo.}{décimo sétimo.}{décimo terceiro.}{décimo.}{oitavo.} \\



\quest{1,5}  Um reservatório de água com capacidade para 20 mil litros encontra-se com 5 mil litros de água num instante inicial $(t)$ igual a zero, em que são abertas duas torneiras. A primeira delas é a única maneira pela qual a água entra no reservatório, e ela despeja $10l$ de água por minuto; a segunda é a única maneira de a água sair do reservatório. A razão entre a quantidade de água que entra e a que sai, nessa ordem, é igual a $\frac{5}{4}$. Considere que $Q(t)$ seja a expressão que
indica o volume de água, em litro, contido no reservatório no instante $t$, dado em minuto, com $t$ variando de $0$ a $7.500$.\\ 
A expressão algébrica para $Q(t)$ é 

\opt{$5.000+2t$}{$5.000-8t$}{$5.000-2t$}{$5.000+10t$}{$5.000-2,5t$}


\quest{2,0} Duas torneiras são utilizadas para encher um tanque vazio. Sabendo que sozinhas elas levam 10 horas e 15 horas, respectivamente, para enchê-lo. Quanto tempo às duas torneiras juntas levam para encher o tanque?\\

\opt{6 horas.}{12 horas e 30 minutos.}{25 horas.}{8 horas e 15 minutos.}{3 horas}


\quest{2,0} 

\opt{}{}{}{}{}

\end{document}
